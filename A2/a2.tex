%%%%%%%%%%%%%%%%%%%%%%%%%%%%%%%%%%%%%%%%%
% Short Sectioned Assignment
% LaTeX Template
% Version 1.0 (5/5/12)
%
% This template has been downloaded from:
% http://www.LaTeXTemplates.com
%
% Original author:
% Frits Wenneker (http://www.howtotex.com)
%
% License:
% CC BY-NC-SA 3.0 (http://creativecommons.org/licenses/by-nc-sa/3.0/)
%
%%%%%%%%%%%%%%%%%%%%%%%%%%%%%%%%%%%%%%%%%

%----------------------------------------------------------------------------------------
%	PACKAGES AND OTHER DOCUMENT CONFIGURATIONS
%----------------------------------------------------------------------------------------

\documentclass[paper=a4, fontsize=11pt]{scrartcl} % A4 paper and 11pt font size

\usepackage[T1]{fontenc} % Use 8-bit encoding that has 256 glyphs
\usepackage{fourier} % Use the Adobe Utopia font for the document - comment this line to return to the LaTeX default
\usepackage[english]{babel} % English language/hyphenation
\usepackage{amsmath,amsfonts,amsthm} % Math packages

\usepackage{lipsum} % Used for inserting dummy 'Lorem ipsum' text into the template

\usepackage{sectsty} % Allows customizing section commands
\allsectionsfont{\centering \normalfont\scshape} % Make all sections centered, the default font and small caps

\usepackage{fancyhdr} % Custom headers and footers
\pagestyle{fancyplain} % Makes all pages in the document conform to the custom headers and footers
\fancyhead{} % No page header - if you want one, create it in the same way as the footers below
\fancyfoot[L]{} % Empty left footer
\fancyfoot[C]{} % Empty center footer
\fancyfoot[R]{\thepage} % Page numbering for right footer
\renewcommand{\headrulewidth}{0pt} % Remove header underlines
\renewcommand{\footrulewidth}{0pt} % Remove footer underlines
\setlength{\headheight}{13.6pt} % Customize the height of the header


\setlength\parindent{0pt} % Removes all indentation from paragraphs - comment this line for an assignment with lots of text

%----------------------------------------------------------------------------------------
%	TITLE SECTION
%----------------------------------------------------------------------------------------

\newcommand{\horrule}[1]{\rule{\linewidth}{#1}} % Create horizontal rule command with 1 argument of height

\title{	
\normalfont \normalsize 
\textsc{CPSC 449: Programming Paradigms} \\ [25pt] % Your university, school and/or department name(s)
\horrule{0.5pt} \\[0.4cm] % Thin top horizontal rule
\huge Assignment Two \\ % The assignment title
\horrule{2pt} \\[0.5cm] % Thick bottom horizontal rule
}

\author{HUNG, Michael; UCID 10099049} % Your name

\date{\normalsize\today} % Today's date or a custom date

\begin{document}

\maketitle % Print the title

%----------------------------------------------------------------------------------------
%	PROBLEM 1
%----------------------------------------------------------------------------------------

To prove or (match x xs) = elem x xs, we must prove the base case and induction step, which are as follows: \\

\textbf{Base Case.} $P$([ ]) = or (match x [ ]) = elem x [ ]) \\
\textbf{Induction Step.} $P$(xs) $\Rightarrow$ $P$(x:xs) \\
or (match x xs) = elem x xs $\Rightarrow$ or (match x (y:ys)) = elem x (y:ys)\\
\\

\textbf{\underline{Base Case}}

or (match x [ ]) = elem x [ ])

or [ ] = elem x [ ] \hspace{50pt}\textit{by match.1}

False = elem x [ ] \hspace{46pt} \textit{by or.1}

False = False \hspace{64pt} \textit{by elem.1}
\\
The base case holds.
\\
\\
We assume that or (match x xs) = elem x xs holds, and try to prove that or (match x (y:ys)) = elem x (y:ys) holds. \\
\\
\textbf{\underline{RHS:}}

elem x (y:ys)\\
(x == y) || (elem x ys) \hspace{30pt} \textit{by elem.2} \\

\textbf{\underline{LHS:}}

or (match x (y:ys)) \\
or ((x == y):(match x ys)) \hspace{12pt} \textit{by match.2} \\
(x == y) || (or (match x ys)) \hspace{12pt} \textit{by or.2} \\

Since in both the LHS and RHS, (x == y) is one of the disjuncts, we need only focus on the second parts,(or (match x ys)) and (elem x ys). However, by our hypothesis, we assumed that or (match x xs) = elem x xs holds. In other words, by the hypothesis, we can change either the LHS to match the RHS, or the RHS to match the LHS. The proof is complete.

\begin{flushright}
$\square$
\end{flushright}

%----------------------------------------------------------------------------------------

\end{document}